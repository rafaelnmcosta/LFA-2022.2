\documentclass[12pt]{article}

% ----------------------------------------- Dados do discente
% Insira os seus dados e do exercício escolhido:
\def\discente{Rafael Nunes Moreira Costa}
\def\matricula{202107855}
\def\ua{17}
\def\up{16}
\def\myling{{7}} % Informe o número da linguagem selecionada.

% ------------------------------------------ Babel & Geometry
\usepackage[brazil]{babel}
\usepackage[T1]{fontenc}
\usepackage[utf8]{inputenc}
\usepackage[a4paper,top=1.0cm,bottom=2.0cm,left=1.5cm,right=1.5cm]{geometry}
%
\usepackage{marvosym} % \Smiley
\usepackage{xcolor}
\usepackage{enumitem}
\usepackage{amsmath,amsfonts,amssymb,mathtools}
\usepackage[breakable]{tcolorbox}
%
\usepackage{tikz}
\usetikzlibrary{automata,arrows,positioning,shapes}
\tikzset{
 node distance=3cm,
 initial text={$P_{\myling}$},
 double distance=1pt,
 every state/.style={semithick,fill=blue!20!white,minimum size=20pt,inner sep=0pt},
 every edge/.style={draw,->,>=stealth,auto,semithick,font=\ttfamily\small}
}
%
% -------------------------------------------------- New commands
%\newcommand{\tvs}{{\large\bf \textvisiblespace}}
\newcommand\vs[1][.8em]{% Visible space!
 \makebox[#1]{%
  \kern.07em
  \vrule height.5ex
  \hrulefill
  \vrule height.5ex
  \kern.07em
 }
}
\newcommand{\ve}{\ensuremath{\varepsilon}}

\begin{document}
% ------------------------------------------------- Cabeçalho
 \begin{tcolorbox}[rounded corners, colback=blue!3, colframe=blue!40!black]
  \footnotesize\textbf{Universidade Federal de Goiás -- UFG}\hfill \textsc{Linguagens Formais e Autômatos -- 2022/2}\\
  \footnotesize\textbf{Instituto de Informática -- INF\hfill Prof. Humberto J. Longo} -- \scriptsize\texttt{longo@inf.ufg.br}
 \end{tcolorbox}\bigskip
%
% ------------------------------------------------- Atividade
\begin{tcolorbox}[rounded corners, colback=blue!2, colframe=blue!40!black, title=\textbf{Atividade AA-\ua}]
 Nesta tarefa deve ser escolhida \textbf{uma} das seguintes opções:
 \begin{enumerate}
  \item Propor uma gramática livre de contexto $G_n$ que gere as cadeias da linguagem selecionada e construir, segundo um dos algoritmos apresentados nas aulas, um PDA que reconheça as cadeias da linguagem gerada pela gramática.
  \item A partir do PDA $P_n$ obtido na atividade ``AA-16 : Autômatos com pilha'', obter uma gramática que gere a linguagem aceita pelo PDA.
 \end{enumerate}
 (Cada aluna(o) deve consultar na descrição da atividade AA--\up, na disciplina INF0333A da plataforma Turing, qual é a linguagem associada ao seu número de matrícula. A descrição da linguagem está disponível no arquivo ``Lista de linguagens livres de contexto'' da Seção ``Coletânea de exercícios''.)
\end{tcolorbox}\bigskip

%
% ------------------------------------ Resolução do exercício
%=========================================================================
\begin{tcolorbox}[rounded corners, colback=yellow!5, colframe=red!40!black, title={\discente\ (\matricula)}]
\begin{itemize}%[before=\color{blue}]
  \item $\mathcal{L}_{\myling} = \{ w=uv^Rv \mid u \in \sum^*, v \in \sum^+ \}$.
%- - - - - - - - - - - - - - - - - - - - - - - - - - - - - - - - 
  \item Gramática $G_{\myling}$ que gera as cadeias da linguagem $\mathcal{L}_{\myling}$:\\ $G_{\myling}=(V,\Sigma,P,S)=(\{A,B,S\},\{0,1\},P,S)$, com
  $
   P =
   \left\{\begin{array}{l}
    S \to AB,\\
    A \to 0A \mid 1A \mid \ve,\\
    B \to 0B0 \mid 1B1 \mid 00 \mid 11
   \end{array}\right\}.
  $
%- - - - - - - - - - - - - - - - - - - - - - - - - - - - - - - - 
  \item PDA $P_{\myling}$ que reconhece as cadeias da linguagem $\mathcal{L}_{\myling}$:
   \begin{center}
   \begin{tikzpicture}[
     accept/.style={accepting,fill=green!20!white},
     initial text={$P_\myling$},
   %  transform shape,
   %  scale=0.95
    ]
     \node[initial,state]            (s0) {$s_0$};
  \node[state,right of=s0]        (s1) {$s_1$};
  \node[state,right of=s1]        (s2) {$s_2$};
  \node[state,right of=s2]        (s3) {$s_3$};
  \node[state,right of=s3,accept] (s4) {$s_4$};
  
  \path (s0) edge node {$\ve,\ve\to \$$} (s1)
        (s1) edge node {$\ve, \ve \to \ve$} (s2)
        (s2) edge node {$\ve, \ve \to \ve$} (s3)
        (s3) edge node {$\ve, \$ \to \ve$} (s4)
        (s1) [loop above] edge node {$
                                        \begin{aligned}
                                        0,\ve \to \ve \\
                                        1, \ve \to \ve
                                        \end{aligned}
                                    $} (s1)
        (s2) [loop above] edge node {$
                                        \begin{aligned}
                                        0,\ve \to 0 \\
                                        1,\ve \to 1
                                        \end{aligned}
                                    $} (s2)
        (s3) [loop above] edge node {$
                                        \begin{aligned}
                                        0,0 \to \ve \\
                                        1,1 \to \ve
                                        \end{aligned}
                                    $} (s3);
   \end{tikzpicture}
   \end{center}
\end{itemize}
\end{tcolorbox}
%=========================================================================
\begin{tcolorbox}[rounded corners, colback=yellow!5, colframe=red!40!black, title={OPÇÂO 1(a): PDA construído a partir da gramática $G_{\myling}$}]
\begin{itemize}
 \item PDA $P^1_{\myling}$, construído a partir da gramática $G_{\myling}$, que reconhece as cadeias da linguagem $\mathcal{L}_{\myling}$:
%  \begin{center}
   \begin{tikzpicture}[
     accept/.style={accepting,fill=green!20!white},
     slopea/.style={sloped,anchor=center,above},
     slopeb/.style={sloped,anchor=center,below},
     initial text={$P_\myling^1$},
     transform shape,
     scale=0.8
    ]
     \node[initial,state]                  (s0) {$s_0$};
     \node[state,right of=s0]              (s1) {$s_1$};
     \node[state,right of=s1]              (s2) {$s_2$};
     \node[state,right of=s2]              (s3) {$s_3$};
     \node[state,right of=s3]              (s4) {$s_4$};
     \node[state,right of=s4,accept]       (s5) {$s_5$};
     \node[state,above of=s2]              (s6) {$s_6$};
     \node[state,below of=s3]        (s7) {$s_7$};
     \node[state,left of=s7]              (s8) {$s_8$};
   
     \path  (s0)
                edge node {$\ve,\ve\to \$$} (s1)
            (s1)
                edge node {$\ve,\ve\to S$}  (s2)
            (s2)
                edge node {
                    $\ve,S\to B$
                } (s3)
                edge [bend left=15] node {
                    $
                    \begin{aligned}
                            1,S\to A\\
                            0,S\to A
                    \end{aligned}
                    $
                } (s6)
            (s6)
                edge [bend left=15] node {
                    $
                    \begin{aligned}
                        \ve, A \to S
                    \end{aligned}
                    $
                } (s2)
            (s3)
                edge node{
                    $
                    \begin{aligned}
                        00,B\to \ve\\
                        11,B\to \ve
                    \end{aligned}
                    $
                } (s4)
                
                edge [loop above] node{
                    $
                    \begin{aligned}
                            1,1\to \ve\\
                            0,0\to \ve
                    \end{aligned}
                    $
                } (s3)
                
                edge node {
                    $
                    \begin{aligned}
                        \ve,B\to 1\\
                        \ve,B\to 0
                    \end{aligned}
                    $
                } (s7)

            (s4)
                edge [loop below] node {
                    $
                    \begin{aligned}
                        1,1\to \ve\\
                        0,0\to \ve
                    \end{aligned}
                    $
                } (s4)

                edge node {
                    $
                        \ve, \$ \to \ve
                    $
                } (s5)
            (s7)
                edge node {
                    $
                        \ve,\ve \to B
                    $
                } (s8)
            (s8)
                edge node {
                    $
                    \begin{aligned}
                        \ve, \ve \to 1\\
                        \ve, \ve \to 0
                    \end{aligned}
                    $
                } (s3)
            ;
   \end{tikzpicture}
%  \end{center}
 \end{itemize}
\end{tcolorbox}
%=========================================================================
\begin{tcolorbox}[rounded corners, colback=yellow!5, colframe=red!40!black, title={OPÇÂO 1(b): PDA construído a partir da forma normal de Greibach da gramática $G_{\myling}$}]
\begin{itemize}
 \item Versão da gramática $G_{\myling}$ na forma normal de Greibach:\\ $G^1_{\myling}=(V,\Sigma,P,S)=(\{S,A,B,C,D\},\{0,1\},P,S)$, com
    $
     P =
     \left\{\begin{array}{r@{\;\to\;}l}
      S   & A, \\
      A   & 0A \mid 1A \mid \ve B,\\
      B   & 0C \mid 1D \mid 00 \mid 11,\\
      C   & B0,\\
      D   & B1\\
     \end{array}\right\}.
    $
%- - - - - - - - - - - - - - - - - - - - - - - - - - - - - - - -
 \item PDA $P^2_{\myling}$ obtido a partir da gramática $G^1_{\myling}$:
   \begin{center}
   \begin{tikzpicture}[
     accept/.style={accepting,fill=green!20!white},
     initial text={$P^2_\myling$},
     transform shape,
     scale=0.95
    ]
        \node[initial,state]            (s0) {$s_0$};
        \node[state,right of=s0]        (s1) {$s_1$};
        \node[state,right of=s1]        (s2) {$s_2$};
        \node[state,right of=s2]        (s3) {$s_3$};
        \node[state,right of=s3,accept] (s4) {$s_4$};
  
  \path (s0) edge node {$\ve,\ve\to A$} (s1)
        (s1) edge node {$\ve, A \to B$} (s2)
        (s2) edge node {$\ve, \ve \to \ve$} (s3)
        (s3) edge node {$\ve, B \to \ve$} (s4)
        (s1) [loop above] edge node {$
                                        \begin{aligned}
                                        0, A \to A \\
                                        1, A \to A
                                        \end{aligned}
                                    $} (s1)
        (s2) [loop above] edge node {$
                                        \begin{aligned}
                                        0,\ve \to C \\
                                        1,\ve \to D
                                        \end{aligned}
                                    $} (s2)
        (s3) [loop above] edge node {$
                                        \begin{aligned}
                                        0,C \to \ve \\
                                        1,D \to \ve
                                        \end{aligned}
                                    $} (s3);
   \end{tikzpicture}
   \end{center}
 \end{itemize}
\end{tcolorbox}
%=========================================================================
%\begin{tcolorbox}[rounded corners, colback=yellow!5, colframe=red!40!black, title={OPÇÂO 2: Gramática $G^2_{\myling}$ obtida a partir do PDA %$P_{\myling}$}]
% \begin{itemize}
%  \item Vide exemplo 7.28 nos \emph{slides} \ldots {\Large \Smiley}
% \end{itemize}
%\end{tcolorbox}
%=========================================================================
\end{document}
%

