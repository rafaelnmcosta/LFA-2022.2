\documentclass[12pt]{article}

% ----------------------------------------- Dados do discente
% Insira os seus dados e do exercício escolhido:
\def\discente{Rafael}
\def\matricula{202107855}
\def\ua{04}
\def\myling{{29}} % Informe o número da linguagem selecionada.

% ------------------------------------------ Babel & Geometry
\usepackage[brazil]{babel}
\usepackage[T1]{fontenc}
\usepackage[utf8]{inputenc}
\usepackage[a4paper,top=0.5cm,bottom=1.5cm,left=1.5cm,right=1.5cm,nohead,nofoot]{geometry}
%
\usepackage{xcolor}
\usepackage{enumitem}
\usepackage{mathtools}
\usepackage{tcolorbox}

\usepackage{tikz}
\usetikzlibrary{automata,arrows,positioning}
\tikzset{
 node distance=2.5cm,
 initial text={$M_\myling$},
 double distance=1pt,
 every state/.style={semithick,fill=blue!20!white,minimum size=20pt,inner sep=0pt},
 every edge/.style={draw,->,>=stealth,auto,semithick,font=\ttfamily\small}
}

%\newcommand{\concatL}{\ensuremath{{\scriptstyle\circ}}}%

\begin{document}
% ------------------------------------------------- Cabeçalho
 \begin{tcolorbox}[rounded corners, colback=blue!3, colframe=blue!40!black]
  \footnotesize\textbf{Universidade Federal de Goiás -- UFG}\hfill \textsc{Linguagens Formais e Autômatos -- 2022/2}\\
  \footnotesize\textbf{Instituto de Informática -- INF\hfill Prof. Humberto J. Longo} -- \scriptsize\texttt{longo@inf.ufg.br}
 \end{tcolorbox}\bigskip
%
% ------------------------------------------------- Atividade
\begin{tcolorbox}[rounded corners, colback=blue!2, colframe=blue!40!black, title=\textbf{Atividade AA-\ua}]
   Nesta tarefa deve-se propôr um autômato finito determinístico (DFA) que reconheça as cadeias da linguagem selecionada. Especifique a tupla que define o DFA e desenhe o correspondente diagrama de estados. (Cada aluna(o) deve consultar na descrição da atividade AA-\ua, na disciplina INF0333A da plataforma Turing, qual é a linguagem associada ao seu número de matrícula. A descrição da linguagem está disponível no arquivo ``Lista de linguagens regulares'' da Seção ``Coletânea de exercícios''.)
\end{tcolorbox}\bigskip
%
% ------------------------------------ Resolução do exercício
\begin{tcolorbox}[rounded corners, colback=yellow!5, colframe=red!40!black, title=\textbf{\matricula\ -- \discente}]
 \begin{itemize}[leftmargin=*]
% 
  \item $\mathcal{L}_\myling = \{w\mid w$ contém uma, duas, ou três ocorrências do símbolo $0$$\}$.
%
  \item Autômato finito determinístico que reconhece as cadeias da linguagem  $\mathcal{L}_\myling$:\\
    $M_\myling=\langle\Sigma=\{0,1\},S=\{s_0,s_1,s_2,s_3,s_4\},s_0,\delta,F=\{s_2,s_3\}\rangle$, com a função $\delta$ definida por:\\
    $$\begin{array}{|c|cc|}
     \hline
     \delta & 0   & 1\\
     \hline
        s_0 & s_1 & s_0\\
        s_1 & s_2 & s_1\\
        s_2 & s_3 & s_2\\
        s_3 & s_4 & s_3\\
        s_4 & s_4 & s_4\\
     \hline
    \end{array}$$
  \begin{center}
   \begin{tikzpicture}[
    %node distance=2.cm,
     accept/.style={accepting,fill=green!20!white},
     reject/.style={fill=red!20!white},
%     transform shape,
%     scale=1.2
    ]
     \node[state,initial]            (s0) {$s_0$};
     \node[state,right of=s0,accept] (s1) {$s_1$};
     \node[state,right of=s1,accept] (s2) {$s_2$};
     \node[state,right of=s2,accept] (s3) {$s_3$};
     \node[state,right of=s3,reject] (s4) {$s_4$};
   %
     \draw (s0) edge               node {0}   (s1)
                edge[loop below]   node {1}   (s0)
           (s1) edge               node {0}   (s2)
                edge[loop below]   node {1}   (s1)
           (s2) edge               node {0}   (s3)
                edge[loop below]   node {1}   (s2)
           (s3) edge               node {0}   (s4)
                edge[loop below]   node {1}   (s3)
           (s4) edge[loop below]   node {0,1} (s4)
   \end{tikzpicture}
  \end{center}
 \end{itemize}
\end{tcolorbox}
%
%-----------------------------------------------
\end{document}
%
