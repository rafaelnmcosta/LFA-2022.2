\documentclass[12pt]{article}

% ----------------------------------------- Dados do discente
% Insira os seus dados e do exercício escolhido:
\def\discente{Rafael Nunes Moreira Costa}
\def\matricula{202107855}
\def\ua{08}
\def\myling{{26}} % Informe o número da linguagem selecionada.

% ------------------------------------------ Babel & Geometry
\usepackage[brazil]{babel}
\usepackage[T1]{fontenc}
\usepackage[utf8]{inputenc}
\usepackage[a4paper,top=0.5cm,bottom=1.5cm,left=1.5cm,right=1.5cm,nohead,nofoot]{geometry}
%
\usepackage{xcolor}
\usepackage{enumitem}
\usepackage{amsmath,amsfonts,amssymb,mathtools}
\usepackage[breakable]{tcolorbox}
\usepackage{booktabs}

\usepackage{tikz}
\usetikzlibrary{automata,shapes,arrows,positioning}
\tikzset{
 node distance=2.5cm,
 initial text={$M_{\mathcal{L}_\myling}$},
 double distance=1pt,
 every state/.style={semithick,fill=blue!20!white,minimum size=20pt,inner sep=0pt},
 every edge/.style={draw,->,>=stealth,auto,semithick,font=\ttfamily\small},
 accept/.style={accepting,fill=green!20!white},
 reject/.style={fill=red!20!white},
 slopea/.style={sloped,anchor=center,above},
 slopeb/.style={sloped,anchor=center,below}
}

%=-=-=-=-=-=-=-=-=-=-=-=-=-=-=-=-=-= N e w   c o m m a n d s
\newcommand{\ve}{\ensuremath{\varepsilon}}

\begin{document}
% ------------------------------------------------- Cabeçalho
 \begin{tcolorbox}[rounded corners, colback=blue!3, colframe=blue!40!black]
  \footnotesize\textbf{Universidade Federal de Goiás -- UFG}\hfill \textsc{Linguagens Formais e Autômatos -- 2022/2}\\
  \footnotesize\textbf{Instituto de Informática -- INF\hfill Prof. Humberto J. Longo} -- \scriptsize\texttt{longo@inf.ufg.br}
 \end{tcolorbox}\bigskip
%
% ------------------------------------------------- Atividade
\begin{tcolorbox}[rounded corners, colback=blue!2, colframe=blue!40!black, title=\textbf{Atividade AA-\ua}]
  Nesta tarefa deve-se ($i$) propôr um autômato finito determinístico \textbf{mínimo} $D$ que reconheça as cadeias da linguagem selecionada, a partir de $D$ construir uma gramática $G_1$ que gere as cadeias reconhecidas por $D$ e a partir de $G_1$, usando o algoritmo baseado em sistemas de equações, extrair uma expressão regular $\mathcal{R}_1$ que gere as mesmas cadeias geradas por $G_1$; ($ii$) propôr um autômato finito não-determinístico $N$ que reconheça as cadeias da linguagem selecionada e, como no item ($i$), obter a partir de $N$ uma gramática $G_2$ (não necessariamente será regular!) e, a partir desta, obter uma expressão regular $\mathcal{R}_2$. O autômato $N$ pode ser um NFA ou NFA-$\varepsilon$, com pelo menos uma transição não determinística ou uma transição $\varepsilon$. \textbf{Atenção:} NFA's criados a partir do simples acréscimo de transições $\delta(s_i,\varepsilon)=s_i$ ($\varepsilon$-laços) a um DFA não serão considerados corretos, por não permitirem uma avaliação razoável do aprendizado dos conceitos abordados nesta atividade avaliativa. (Cada aluna(o) deve consultar na descrição da atividade AA--\ua, na disciplina INF0333A da plataforma Turing, qual é a linguagem associada ao seu número de matrícula. A descrição da linguagem está disponível no arquivo ``lista de linguagens regulares'' da Seção ``Coletânea de exercícios''.)
\end{tcolorbox}\bigskip

%
% ------------------------------------ Resolução do exercício
\begin{tcolorbox}[rounded corners, colback=yellow!5, colframe=red!40!black, title={\discente\ (\matricula)}]
 \begin{itemize}[leftmargin=*]
% - - - - - - - - - - - - - - - - - - - - - - - - - - - - - - - - - - - -
  \item $\mathcal{L}_\myling = \{w \mid |w|_0 + |w|_1 = 2k+1, k \in \mathbb{N}$ \textbf{e w não contém 10}. $\}$
% - - - - - - - - - - - - - - - - - - - - - - - - - - - - - - - - - - - -
  \item $ER(\mathcal{L}_\myling) = 0(00\cup11)\cup1(11)^*$.
% - - - - - - - - - - - - - - - - - - - - - - - - - - - - - - - - - - - -
 \end{itemize}
\end{tcolorbox}\bigskip
%=========================================================================
\begin{tcolorbox}[rounded corners, colback=yellow!5, colframe=red!40!black, title={DFA mínimo que reconhece as cadeias de $\mathcal{L}_\myling$}]
 \centering
 \begin{tikzpicture}[
%    transform shape,
%    scale=0.8
 ]
  \node[state,initial]            (s0) {$S$};
  \node[state,below right of=s0, accept]        (s1) {$A$};
  \node[state,right of=s1]        (s2) {$B$};
  \node[state,above right of=s0,accept] (s3) {$C$};
  \node[state,right of=s3] (s4) {$D$};
%
  \draw (s0) edge               node {0}   (s1)
             edge               node {1}   (s3)
        (s1) edge[bend left=25] node {0}   (s2)
             edge[bend left=13] node {1}   (s4)
        (s2) edge[bend left=25] node {0}   (s1)
             edge[bend right=13] node {1}   (s3)
        (s3) edge[bend left=25] node {1}   (s4)
        (s4) edge[bend left=25] node {1}   (s3);
 \end{tikzpicture}
\end{tcolorbox}\bigskip
%=========================================================================
\begin{tcolorbox}[breakable,rounded corners, colback=yellow!5, colframe=red!40!black, title={Gramática $G_1$ que gera as cadeias de $\mathcal{L}_\myling$}]
 \begin{align*}
 G_1 &= (V,\Sigma,P,S)=(\{A,B,C,D,S\},\{0,1\},P,S),\text{ com:}\\
   P &=
   \left\{\begin{array}{l}
    S \to 0A \mid 1C,\\
    A \to 0B \mid 1D \mid \varepsilon,\\
    B \to 0A \mid 1C,\\
    C \to 1D \mid \varepsilon,\\
    D \to 1C
   \end{array}\right\}.
 \end{align*}
\end{tcolorbox}\bigskip
%=========================================================================
\begin{tcolorbox}[breakable,rounded corners, colback=yellow!5, colframe=red!40!black, title={Extração de expressão regular $\mathcal{R}_1$ da gramática $G$, tal que $\mathcal{L}(\mathcal{R}_1)=\mathcal{L}(G_1)$}]
  $$
  \begin{array}{ccll}
  \toprule
  \textbf{Etapa} & \textbf{Eq.} & \textbf{Expressão} & \textbf{Ação}\\
  
  \midrule
    I & 1 & S = 0A \cup 1C           &\\
      & 2 & A = 0B \cup 1D \cup \ve  &\\
      & 3 & B = 0A \cup 1C           &\\
      & 4 & C = 1D \cup \ve          &\\
      & 5 & D = 1C                   &\\
  \midrule
   II & 1 & S = 0A \cup 1C           &\\
      & 2 & A = 0B \cup 11C \cup \ve & I.5 \to I.2 \\
      & 3 & B = 0A \cup 1C           &\\
      & 4 & C = 11C \cup \ve & I.5 \to I.4         \\
  \midrule
  III & 1 & S = 0A \cup 1C           &\\
      & 2 & A = 0B \cup 11C \cup \ve &\\
      & 3 & B = 0A \cup 1C           &\\
      & 4 & C = (11)^*(\ve) = (11)^* & II.4 \to \texttt{Lema de Arden} \\
  \midrule
  IV  & 1 & S = 0A \cup 1(11)^*           & III.4 \to III.1\\
      & 2 & A = 0B \cup 11(11)^* \cup \ve & III.4 \to III.2\\
      & 3 & B = 0A \cup 1(11)^*           & III.4 \to III.3\\
  \midrule
   V  & 1 & S = 0A \cup 1(11)^*           &\\
      & 2 & A = 00A \cup 01(11)^* \cup 11(11)^* \cup \ve & IV.3 \to IV.2, \texttt{Fatoração}\\
  \midrule
   VI & 1 & S = 0A \cup 1(11)^*           &\\
      & 2 & A = (00)^*(01(11)^* \cup 11(11)^* \cup \ve) & V.2 \to \texttt{Lema de Arden}\\
  \midrule
  VII & 1 & S = 0((00)^*(01(11)^*) \cup 11(11)^* \cup \ve)) \cup 1(11)^* & VI.2 \to VI.1\\
  \midrule
 VIII & 1 & S = 0((00)^*(01(11)^*)) \cup 0(11)^+ \cup 0(\ve) \cup 1(11)^* & VI.1 \to \texttt{Fatoração}\\
  \bottomrule
  \end{array}
  $$
\end{tcolorbox}\bigskip
%=========================================================================
\begin{tcolorbox}[rounded corners, colback=yellow!5, colframe=red!40!black, title={NFA que reconhece as cadeias de $\mathcal{L}_\myling$}]
 \centering
 \begin{tikzpicture}
  [
   initial text={$N_\myling$},
   accept/.style={accepting,fill=green!20!white},
   reject/.style={fill=red!20!white},
%    node distance=2.cm,
%    transform shape,
%    scale=1.2
  ]
  \node[state,initial]            (s) {$S$};
  \node[state,right of=s, accept]        (a) {$A$};
  \node[state,right of=a]        (b) {$B$};
  \node[state,below of=s,accept] (c) {$C$};
  \node[state,right of=c] (d) {$D$};
  %
    \draw (s) edge                       node {0,1}             (a)
              edge                       node {0}   (c)
          (a) edge          node {1}               (b)
          (b) edge [bend left=25] node {1} (a)
          (c) edge [bend left=25] node {0} (s)
              edge node {1} (d)
          (d) edge node {1} (a);
   \end{tikzpicture}
\end{tcolorbox}\bigskip
%=========================================================================
\begin{tcolorbox}[rounded corners, colback=yellow!5, colframe=red!40!black, title={Gramática $G_2$ que gera as cadeias da linguagem $\mathcal{L}_{\myling}$}]
 \begin{align*}
 G_2 &=(V,\Sigma,P,S)=(\{A,B,C,D\},\{0,1\},P,S), \text{ com}\\
   P &=
   \left\{\begin{array}{l}
    S \to 0A \mid 0C \mid 1A,\\
    A \to 1B \mid \ve,\\
    B \to 1A,\\
    C \to 0S \mid 1D \mid \ve,\\
    D \to 1A\\
   \end{array}\right\}.
 \end{align*}
\end{tcolorbox}\bigskip
%%=========================================================================
\begin{tcolorbox}[breakable,rounded corners, colback=yellow!5, colframe=red!40!black, title={Extração de expressão regular $\mathcal{R}_2$ da gramática $G_2$, tal que $\mathcal{L}(\mathcal{R}_2)=\mathcal{L}(G_2)$}]
  $$
  \begin{array}{ccll}
  \toprule
  \textbf{Etapa} & \textbf{Eq.} & \textbf{Expressão} & \textbf{Ação}\\
  \midrule
    I & 1 & S = 0A \cup 0C \cup 1A        &\\
      & 2 & A = 1B \cup \ve               &\\
      & 3 & B = 1A                        &\\
      & 4 & C = 0S \cup 1D \cup \ve       &\\
      & 5 & D = 1A                       &\\
  \midrule
   II & 1 & S = 0(1B \cup \ve)\ \cup \ 0C \ \cup \ 1(1B \cup \ve) & I.2 \to I.1,\ \text{Fatoração}\\
      & 2 & B = 1(1B \cup \ve) = 11B \cup 1  & I.2 \to I.3,\ \text{Fatoração}\\
      & 3 & C = 0S \cup 1D \cup \ve       &\\
      & 4 & D = 1(1B \cup \ve) = 11B \cup 1  & I.2 \to I.5,\ \text{Fatoração}\\
  \midrule
  III & 1 & S = (01B \cup 0)\ \cup \ 0C \ \cup \ (11B \cup 1) & \text{Fatoração}\\
      & 2 & B = (11)^*1             & II.2 \to \text{Lema de Arden}\\
      & 3 & C = 0S \cup 1D \cup \ve       &\\
      & 4 & D = 11B \cup 1            & \\
  \midrule
  IV  & 1 & S = 01(11)^*1\cup 010\ \cup \ 0C \ \cup \ 11(11)^*1\cup 1 & III.2 \to III.1\\
      & 2 & C = 0S \cup 1D \cup \ve       &\\
      & 3 & D = 11(11)^*1 \cup 1            & III.2 \to III.4\\
  \midrule
  V   & 1 & S = 01(11)^*1\cup 010\ \cup \ 0C \ \cup \ 11(11)^*1\cup 1 &\\
      & 2 & C = 0S \ \cup \ 111(11)^*1 \cup 11 \ \cup \ \ve       & IV.3 \to IV.2 \ \text{Fatoração}\\
  \midrule
  VI  & 1 & \begin{tabular}{c c} 
       $ S = 01(11)^*1\cup 010\ \cup \ 00S \ \cup $
       \\ $ 0111(11)^*1 \cup 011 \cup 0 \ \cup \ 11(11)^*1 \cup 1$
       \end{tabular} 
       & V.2 \to V.1 \ \text{Fatoração}\\
  \midrule
  VII & 1 & \begin{tabular}{c c} 
        $ S = ( 01(11)^*1\cup 010 \cup 00 )^* $
        \\ $ 0111(11)^*1 \cup 011 \cup 0 \cup 11(11)^*1\cup 1 $
        \end{tabular}
       & VI.1 \to \ \text{Lema de Arden}\\
  \bottomrule
  \end{array}
  $$
\end{tcolorbox}
%=========================================================================
\end{document}
%
