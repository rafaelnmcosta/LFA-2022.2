\documentclass[12pt]{article}

% ----------------------------------------- Dados do discente
% Insira os seus dados e do exercício escolhido:
\def\discente{Rafael Nunes Moreira Costa}
\def\matricula{202107855}
\def\ua{02}
\def\myling{{12}} % Informe o número da linguagem selecionada.

% ------------------------------------------ Babel & Geometry
\usepackage[brazil]{babel}
\usepackage[T1]{fontenc}
\usepackage[utf8]{inputenc}
\usepackage[a4paper,top=0.5cm,bottom=1.5cm,left=1.5cm,right=1.5cm,nohead,nofoot]{geometry}
%
\usepackage{xcolor}
\usepackage{enumitem}
\usepackage{mathtools}
\usepackage{tcolorbox}

\newcommand{\concatL}{\ensuremath{{\scriptstyle\circ}}}%

\begin{document}
% ------------------------------------------------- Cabeçalho
 \begin{tcolorbox}[rounded corners, colback=blue!3, colframe=blue!40!black]
  \footnotesize\textbf{Universidade Federal de Goiás -- UFG}\hfill \textsc{Linguagens Formais e Autômatos -- 2022/2}\\
  \footnotesize\textbf{Instituto de Informática -- INF\hfill Prof. Humberto J. Longo} -- \scriptsize\texttt{longo@inf.ufg.br}
 \end{tcolorbox}\bigskip
%
% ------------------------------------------------- Atividade
\begin{tcolorbox}[rounded corners, colback=blue!2, colframe=blue!40!black, title=\textbf{Atividade AA-\ua}]
   Nesta tarefa deve-se propôr uma definição por \textbf{Conjuntos Regulares} para a linguagem selecionada. (Cada aluno deve consultar na descrição da atividade AA-\ua, na disciplina INF0333A da plataforma Turing, qual é a linguagem associada ao seu número de matrícula. A descrição da linguagem está disponível no arquivo ``Lista de linguagens regulares'' da Seção ``Coletânea de exercícios''.)
\end{tcolorbox}\bigskip
%
% ------------------------------------ Resolução do exercício
%
\begin{tcolorbox}[rounded corners, colback=yellow!5, colframe=red!40!black, title=\textbf{\matricula\ -- \discente}]
 \begin{itemize}[leftmargin=*]
  \item $\mathcal{L}_\myling = \{w\in\{0,1\}^*\mid w$ não contém $101$ e termina com $1\}$.
%
  \item Definição, por operações com conjuntos regulares, das palavras pertencentes à linguagem $\mathcal{L}_\myling$:
   $$\mathcal{CR}(\mathcal{L}_\myling) =
% * * * * * * * * * * * * * * * * * * * * * * 
    \{0\}^*\concatL(\{001, 100\}^*\cup\{11\}^*\cup \{00\}^*)^*\concatL\{1\}^*
% * * * * * * * * * * * * * * * * * * * * * * 
   .$$
 \end{itemize}
\end{tcolorbox}
%
%% MAIS UM EXEMPLO!
%\begin{tcolorbox}[rounded corners, colback=yellow!5, colframe=red!40!black, title=\textbf{\matricula\ -- \discente}]
% \begin{itemize}[leftmargin=*]
%% 
%  \item $\mathcal{L}_\myling = \{w\in\{0,1\}^*\mid w=0u1$ ou $w=1u0$, com $u\in\Sigma^*\}$.
%%
%  \item Definição de $\mathcal{L}_\myling$ por conjuntos regulares:
%   $$\mathcal{CR}(\mathcal{L}_\myling) =
%% * * * * * * * * * * * * * * * * * * * * * * 
%    \{0\}\concatL\{0,1\}^*\concatL\{1\}\cup\{1\}\concatL\{0,1\}^*\concatL\{0\}
%% * * * * * * * * * * * * * * * * * * * * * * 
%   .$$
% \end{itemize}
%\end{tcolorbox}
%
%-----------------------------------------------
\end{document}
%
